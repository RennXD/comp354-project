\documentclass[12pt]{article}
\usepackage{color}
\pagestyle{empty}
\setcounter{secnumdepth}{2}

\topmargin=0cm
\oddsidemargin=0cm
\textheight=22.0cm
\textwidth=16cm
\parindent=0cm
\parskip=0.15cm
\topskip=0truecm
\raggedbottom
\abovedisplayskip=3mm
\belowdisplayskip=3mm
\abovedisplayshortskip=0mm
\belowdisplayshortskip=2mm
\normalbaselineskip=12pt
\normalbaselines

\begin{document}

\vspace*{0.5in}
\centerline{\bf\Large Requirements Document}

\vspace*{0.5in}
\centerline{\bf\Large Team PA-PK}

\vspace*{0.5in}
\centerline{\bf\Large 17 January 2012}

\vspace*{1.5in}
\begin{table}[htbp]
\caption{Team}
\begin{center}
\begin{tabular}{|r | c|}
\hline
Name & ID Number \\
\hline\hline
Anne-Laure Ehresmann & 27858906 \\
\hline
\end{tabular}
\end{center}
\end{table}

\tableofcontents
\listoffigures

\clearpage

\section{System}\subsection{Purpose}

The purpose of this document is to define requirements for the  desktop application myMoney.
There exists a plethora of software for money management, each greatly varying in design due to the complex and multifarious clientele. This document may thus be to orient the development of the application. It may be used by:

\begin{table}[htbp]
\caption{Document Users}
\begin{center}
\begin{tabular}{|c|p{10cm}|}
\hline
Users and customers       & To give feedback about the requirements. \\
\hline
System developers         & To understand what functions and properties the system must contain. \\
\hline
Testers                   & To test the system against the requirements. \\
\hline
Writers of user manuals   & To get material for user manuals. \\
\hline
Project team              & To follow-up the status of the project against the requirements. \\
\hline
\end{tabular}
\end{center}
\end{table}


\subsection{Business Goals}

\section{Domain Concepts}

\section{Actors}

\section{Use Cases}

\subsection{Overview}

\begin{figure}[htbp]
%insert diagram here
\caption{Use Case Diagram}
\label{fig:use-case-diagram}
\end{figure}

\subsubsection{Use Case 1} \label{uc:1}

\noindent
{\bf Name}\\
Give a name.

\noindent
{\bf Summary}\\
A short summary/description/story.

\noindent
{\bf Actors}\\

\noindent
{\bf Precondition}\\

\noindent
{\bf Main Scenario}\\
\vspace*{-0.2in}
\begin{enumerate}
\item Describe step 1.
\item Describe step 2.
\item Describe step 3.
\end{enumerate}

\noindent
{\bf Exceptions}\\

\noindent
{\bf Postcondition}\\

\noindent
{\bf Priority}\\

\noindent
{\bf Traces to Test Cases}\\
Add when test cases done.

\subsubsection{Use Case 2} \label{uc:2}

\section{Non-Functional Constraints}

\section{Data Dictionary}

\section{References}

\appendix

\section{Description of File Format: Tasks}

Describe input file format.

\section{Description of File Format: Persons}

Describe output file format.

\end{document}
