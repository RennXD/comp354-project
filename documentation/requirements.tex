ss[11pt]{article}
\usepackage{color}
\usepackage{enumitem}
\usepackage{booktabs}
\pagestyle{empty}
\setcounter{secnumdepth}{2}

\topmargin=0cm
\oddsidemargin=0cm
\textheight=22.0cm
\textwidth=17cm
\parindent=0cm
\parskip=0.15cm
\topskip=0truecm
\raggedbottom
\abovedisplayskip=3mm
\belowdisplayskip=3mm
\abovedisplayshortskip=0mm
\belowdisplayshortskip=2mm
\normalbaselineskip=12pt
\normalbaselines

% use case stuff
\newcounter{use case ID}

% environment slightly edited from https://tex.stackexchange.com/questions/10293/latex-template-for-use-cases
\newcommand\tabularhead[1]{
\begin{table}[ht]
	\addtocounter{use case ID}{1}
	\caption{Use Case \arabic{use case ID} - #1}
	\vspace{0.2cm}
	\begin{tabular}{|p{0.2\linewidth}|p{0.70\linewidth}|}
	\hline
		\textbf{Action} & \textbf{#1} \\
		\hline}

	\newcommand\addrow[2]{#1 & #2\\ \hline}

	\newcommand\addmulrow[2]{ \begin{minipage}[t][][t]{2.5cm}#1\end{minipage}
		&\begin{minipage}[t][][t]{11cm}
		\begin{enumerate}[itemsep=-1ex] #2   \end{enumerate}
	\end{minipage}\\ \hline}

	\newenvironment{usecase}{\tabularhead}
{\hline\end{tabular}\end{table}}


\begin{document}

\vspace*{0.5in}
\centerline{\bf\Large Requirements Document}

\vspace*{0.5in}
\centerline{\bf\Large Team PA-PK}

\vspace*{0.5in}
\centerline{\bf\Large 17 January 2012}

\vspace*{1.5in}
\begin{table}[htbp]
\caption{Team}
\begin{center}
\begin{tabular}{|r | c|}
\hline
Name & ID Number \\
\hline\hline
Anne-Laure Ehresmann & 27858906 \\
\hline
\end{tabular}
\end{center}
\end{table}

\tableofcontents
\listoffigures
\listoftables

\clearpage

\section{System}\subsection{Purpose}

The purpose of this document is to define requirements for the  desktop application myMoney.
There exists a plethora of software for money management, each greatly varying in design due to the complex and multifarious clientele. This document may thus be to orient the development of the application. It may be used by:

\begin{table}[htbp]
\caption{Document Users}
\begin{center}
\begin{tabular}{|c|p{10cm}|}
\hline
Users and customers       & To give feedback about the requirements. \\
\hline
System developers         & To understand what functions and properties the system must contain. \\
\hline
Testers                   & To test the system against the requirements. \\
\hline
Writers of user manuals   & To get material for user manuals. \\
\hline
Project team              & To follow-up the status of the project against the requirements. \\
\hline
\end{tabular}
\end{center}
\end{table}


\subsection{Business Goals}

\section{Domain Concepts}

\section{Actors}

\section{Use Cases}

\subsection{Overview}

\begin{figure}[htbp]
%insert diagram here
\caption{Use Case Diagram}
\label{fig:use-case-diagram}
\end{figure}

\begin{usecase}{Create user account}
	\addrow{Summary}{User gives information about a new user account, system validates it and creates the account.}
	\addrow{Scope}{money and budget management application}

	\addrow{Level}{user-goal}

	\addrow{Actors}{\textbf{User}}

	\addmulrow{Stakeholders and Interests}{
		\item User: Wants fast and easy account creation, clear and comprehensible display, proof of successful account creation.
		\item Company: Wants user interests to be fulfilled, wants to prevent erroneous input, wants fast communication with the local accountdatabase as well as fault tolerance in case of database conflicts, issues with editing authorisation, or other possible database problems.}
	\addrow{Pre-Conditions}{User has opened the application and is in the startup menu.}
	\addrow{Success Guarantee}{Account successfully saved in local account database, with name and password as specified by the user.}
	\addmulrow{Main Success Flow}{
		\item User enters a username and password.
		\item System validates username and password (format, whether username is already used, etc).
		\item System creates new account.
		\item System notifies the user of the successful account creation, then returns to startup menu.}
	\addrow{Exceptions}{}
	\addrow{PostConditions}{}
	\addrow{Priority}{}
	\addrow{Traces to Test Cases}{}
\end{usecase}



\subsubsection{Use Case 2} \label{uc:2}

\section{Non-Functional Constraints}

\section{Data Dictionary}

\section{References}

\appendix

\section{Description of File Format: Tasks}

Describe input file format.

\section{Description of File Format: Persons}

Describe output file format.

\end{document}
